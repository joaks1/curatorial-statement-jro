\documentclass[10pt]{article}
%\usepackage{anysize}
%\papersize{11in}{8.5in}
%\marginsize{1in}{1in}{.5in}{.5in}
\textwidth = 6.5 in
\textheight = 9 in
\oddsidemargin = 0.0 in
\evensidemargin = 0.0 in
\topmargin = -0.5 in
\headheight = 0.35 in
\headsep = 0.15 in
\topskip = 0 in
\footskip = 0.5 in
\pagenumbering{arabic}
\usepackage{setspace}
\usepackage[usenames]{color}
\usepackage[fleqn]{amsmath}
\usepackage{graphicx}
\usepackage{url}
\usepackage{verbatim}
\usepackage{indentfirst}
\usepackage{booktabs}
\usepackage{multirow}
\usepackage[table]{xcolor}
\usepackage{ragged2e}
\usepackage{xspace}
\usepackage{parskip}
\usepackage{tabulary}
\usepackage[normalem]{ulem}
\usepackage{hyperref}
\hypersetup{pdfborder={0 0 0}, colorlinks=true, urlcolor=blue, linkcolor=black}
\usepackage[format=plain, labelsep=period, justification=raggedright, singlelinecheck=true, skip=2pt, font={footnotesize,sf}, labelfont=bf]{caption}
\usepackage{titlesec}
\usepackage{lastpage}
\usepackage{fancyhdr}
\usepackage{ifthen}
\usepackage{wrapfig}

%\usepackage[round]{natbib}
%\bibliographystyle{evolution}
\usepackage[style=nature]{biblatex}
\bibliography{references}

%% Format headers and footers %%%%%%%%%%%%%%%%%%%%
\pagestyle{fancy}
%\lhead{\ifthenelse{\value{page}=1}{}{\sffamily\footnotesize Jamie Oaks}}
\lhead{\sffamily \emph{\docTitle} \\ Jamie R. Oaks}
%\chead{\ifthenelse{\value{page}=1}{{\scshape \docTitle} \\ Jamie Richard Oaks}{\sffamily\footnotesize \docTitle}}
%\rhead{\ifthenelse{\value{page}=1}{}{\sffamily\footnotesize \today}}
\rhead{\sffamily \today}
\cfoot{\sffamily\footnotesize Page \thepage\ of \pageref{LastPage}}
\renewcommand{\headrulewidth}{0.4pt}
\renewcommand{\footrulewidth}{0pt}

%% Format section titles %%%%%%%%%%%%%%%%%%%%%%%%%
\renewcommand\refname{Peer-reviewed Publications}

\titleformat{\section}[hang]
    {\large\sffamily\bfseries}
    {\S\ \thesection.}{.5em}{}[]
\titlespacing{\section}
    {0mm}{1.0ex plus .1ex minus .1ex}{-0.5ex}

\titleformat{\subsection}[hang]
    {\large\sffamily\itshape}
    {\S\ \thesection.}{.5em}{}[]
\titlespacing{\subsection}
    {0mm}{1.0ex plus .1ex minus .1ex}{-0.5ex}

\titleformat{\subsubsection}[runin]
    {\sffamily\bfseries}
    {\S\ \thesection.}{.5em}{}[.---]
\titlespacing{\subsubsection}
    {\parindent}{0pt}{0pt}
%    {\parindent}{1.0ex plus .1ex minus .1ex}{0pt}

%% Format list environments %%%%%%%%%%%%%%%%%%%%%%%%
\renewcommand{\labelenumii}{\arabic{enumi}.\arabic{enumii}}
\renewcommand{\labelitemi}{$\circ$}

\newenvironment{myEnumerate}{
  \begin{enumerate}
    \setlength{\itemsep}{0.25em}
    \setlength{\parskip}{0pt}
    \setlength{\parsep}{0.5em}}
  {\end{enumerate}}

\newenvironment{myItemize}{
  \begin{itemize}
    \setlength{\leftskip}{-4mm}
    \setlength{\itemsep}{0.25em}
    \setlength{\parskip}{0pt}
    \setlength{\parsep}{0.5em}}
  {\end{itemize}}

%% Basic formatting and spacing %%%%%%%%%%%%%%%%%%%%%
\setlength{\parindent}{0em}
\setlength{\parskip}{0.5em}

%% My functions %%%%%%%%%%%%%%%%%%%%%%%%%%%%%
\newcommand{\ignore}[1]{}
\newcommand{\addTail}[1]{\textit{#1}.---}
\newcommand{\super}[1]{\ensuremath{^{\textrm{#1}}}}
\newcommand{\sub}[1]{\ensuremath{_{\textrm{#1}}}}
\newcommand{\dC}{\ensuremath{^\circ{\textrm{C}}}}
\newcommand{\tableSubItem}{\addtolength{\leftskip}{1em} \labelitemi \xspace}
\newcommand{\myHangIndent}{\hangindent=5mm}

%%%%%%%%%%%%%%%%%%%%%%%%%%%%%%%%%%%%%%%%%%%%%%%%%%%%%%%%%%%%%%
%%%%%%%%%%%%%%%%%%%%%%%%%%%%%%%%%%%%%%%%%%%%%%%%%%%%%%%%%%%%%%
\newcommand{\docTitle}{Curatorial Statement\xspace}
\begin{document}
\raggedright
\singlespacing

My duty as a collection-based scientist is to \textbf{\textit{advocate and
advance the use, maintainence, and value of natural history collections for
understanding the biodiversity of our planet.}}
The challenges of curation are often centered around striking a balance between
the use and long-term preservation of research material.
I believe we are entering a paradigm shift in the ways biodiversity data is
distributed and used by scientist that will greatly reduce the negative impacts
on the longevity of specimens and transform the role of biodiversity
collections.

\section*{The Future of Biodiversity Collections}
In the near future, a typical museum specimen will be associated
with a wealth of digital information that will be available in real-time
to scientists around the world.
Along with the typical location, habitat, and natural history data,
there will be a genome sequence and high-resolution, three-dimensional
digital scans and x-rays for the specimen.
While such a rich digital representation will never supplant the need for
preserving the physical specimen, I believe the data will truly revolutionize
the role of biodiversity collections in science and society.
By greatly reducing the frequency at which specimens need be physically
examined, the longevity of specimens will be extended.
Also, if a specimen is ever damaged or lost, the utility of the specimen will
be largely preserved in its digital form.
More importantly, such rich, real-time data available from millions of
specimens across hundreds of institutions will greatly expand the types of
research that can be addressed with natural history collections.
Such a prominent and modern role will clearly demonstrate the importance
of the collections to the public, funding agencies, and donors.

While this future may seem far off, the collections that are at the forefront
of embracing this shift will balh.
During my experience as the Curatorial Assistant for the Herpetology collections
of the Louisiana State University Museum of Natural Science and University
of Kansas Biodiversity Institute, I became aware that most modern collections
can and should take large steps toward this future.
Many specimens are already associate with a wealth of data.
Many specimens have sequences on
\href{http://www.ncbi.nlm.nih.gov/genbank/}{GenBank}, are part of datasets deposited in
repositories like 
\href{http://datadryad.org}{Dryad},
are tips on trees in \href{http://treebase.org}{TreeBASE}, or better yet,
on the open
\href{http://opentreeoflife.org}{Open Tree of Life}.
Specimens often associated with numerous publications and digital photographs
and audio recordings.
I plan to develop database software for biodiversity collections that allows
all of this data to be easily stored, interfaced, and accessed.
to make our collection databases to be the centralized server of all available
information.
One way to make this a reality is to develop flexible, easy-to-use software
that allows real-time digitization of data as it is collected in the field.
From GPS coordinates, photographs \ldots.
Furthermore, the collection database should automatically be scanning primary databases
such as GenBank, Dryad, PubMed, Web of Knowledge for information derived
from its holdings.

Not only is this important for maximizing the scientific potential of
each sample, but also clearly demonstrates the utility of biodiversity
collections to the public, policy-makers, funding agencies, and donors.

Digitizing in the field in realtime (photographs) to reduce information that is lost and corrupted along way


\section*{Museum-Based Education}
Similar to , 
Similar to curation, technology will greatly enhance museum-based education.

% \section*{Collection-Based Research}
% While my primary research questions focus on specific taxa of Southeast Asia
% (see research statement), an equally important aspect of my work includes
% broad collecting of amphibians and reptiles to document the biodiversiy of
% this relatively poorly known region of the world.
% an important aspect of my work will always be broad biodiversity survey stuff.
% Funding from NSF BSFI Nat Geo

% The SNOMNH Herpetology collection is strong in blah
% My research program would expand holdings of Old World blah, specifically
% Southeast Asia.

\end{document}
%%%%%%%%%%%%%%%%%%%%%%%%%%%%%%%%%%%%%%%%%%%%%%%%%%%%%%%%%%%%%%
%%%%%%%%%%%%%%%%%%%%%%%%%%%%%%%%%%%%%%%%%%%%%%%%%%%%%%%%%%%%%%
