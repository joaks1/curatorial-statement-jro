\documentclass[10pt]{article}
%\usepackage{anysize}
%\papersize{11in}{8.5in}
%\marginsize{1in}{1in}{.5in}{.5in}
\textwidth = 6.5 in
\textheight = 9 in
\oddsidemargin = 0.0 in
\evensidemargin = 0.0 in
\topmargin = -0.5 in
\headheight = 0.35 in
\headsep = 0.15 in
\topskip = 0 in
\footskip = 0.5 in
\pagenumbering{arabic}
\usepackage{setspace}
\usepackage[usenames]{color}
\usepackage[fleqn]{amsmath}
\usepackage{graphicx}
\usepackage{url}
\usepackage{verbatim}
\usepackage{indentfirst}
\usepackage{booktabs}
\usepackage{multirow}
\usepackage[table]{xcolor}
\usepackage{ragged2e}
\usepackage{xspace}
\usepackage{parskip}
\usepackage{tabulary}
\usepackage[normalem]{ulem}
\usepackage{hyperref}
\hypersetup{pdfborder={0 0 0}, colorlinks=true, urlcolor=blue, linkcolor=black}
\usepackage[format=plain, labelsep=period, justification=raggedright, singlelinecheck=true, skip=2pt, font={footnotesize,sf}, labelfont=bf]{caption}
\usepackage{titlesec}
\usepackage{lastpage}
\usepackage{fancyhdr}
\usepackage{ifthen}
\usepackage{wrapfig}

%\usepackage[round]{natbib}
%\bibliographystyle{evolution}
\usepackage[style=nature]{biblatex}
\bibliography{references}

%% Format headers and footers %%%%%%%%%%%%%%%%%%%%
\pagestyle{fancy}
%\lhead{\ifthenelse{\value{page}=1}{}{\sffamily\footnotesize Jamie Oaks}}
\lhead{\sffamily \emph{\docTitle} \\ Jamie R. Oaks}
%\chead{\ifthenelse{\value{page}=1}{{\scshape \docTitle} \\ Jamie Richard Oaks}{\sffamily\footnotesize \docTitle}}
%\rhead{\ifthenelse{\value{page}=1}{}{\sffamily\footnotesize \today}}
\rhead{\sffamily \today}
\cfoot{\sffamily\footnotesize Page \thepage\ of \pageref{LastPage}}
\renewcommand{\headrulewidth}{0.4pt}
\renewcommand{\footrulewidth}{0pt}

%% Format section titles %%%%%%%%%%%%%%%%%%%%%%%%%
\renewcommand\refname{Peer-reviewed Publications}

\titleformat{\section}[hang]
    {\large\sffamily\bfseries}
    {\S\ \thesection.}{.5em}{}[]
\titlespacing{\section}
    {0mm}{1.0ex plus .1ex minus .1ex}{-0.5ex}

\titleformat{\subsection}[hang]
    {\large\sffamily\itshape}
    {\S\ \thesection.}{.5em}{}[]
\titlespacing{\subsection}
    {0mm}{1.0ex plus .1ex minus .1ex}{-0.5ex}

\titleformat{\subsubsection}[runin]
    {\sffamily\bfseries}
    {\S\ \thesection.}{.5em}{}[.---]
\titlespacing{\subsubsection}
    {\parindent}{0pt}{0pt}
%    {\parindent}{1.0ex plus .1ex minus .1ex}{0pt}

%% Format list environments %%%%%%%%%%%%%%%%%%%%%%%%
\renewcommand{\labelenumii}{\arabic{enumi}.\arabic{enumii}}
\renewcommand{\labelitemi}{$\circ$}

\newenvironment{myEnumerate}{
  \begin{enumerate}
    \setlength{\itemsep}{0.25em}
    \setlength{\parskip}{0pt}
    \setlength{\parsep}{0.5em}}
  {\end{enumerate}}

\newenvironment{myItemize}{
  \begin{itemize}
    \setlength{\leftskip}{-4mm}
    \setlength{\itemsep}{0.25em}
    \setlength{\parskip}{0pt}
    \setlength{\parsep}{0.5em}}
  {\end{itemize}}

%% Basic formatting and spacing %%%%%%%%%%%%%%%%%%%%%
\setlength{\parindent}{0em}
\setlength{\parskip}{0.5em}

%% My functions %%%%%%%%%%%%%%%%%%%%%%%%%%%%%
\newcommand{\ignore}[1]{}
\newcommand{\addTail}[1]{\textit{#1}.---}
\newcommand{\super}[1]{\ensuremath{^{\textrm{#1}}}}
\newcommand{\sub}[1]{\ensuremath{_{\textrm{#1}}}}
\newcommand{\dC}{\ensuremath{^\circ{\textrm{C}}}}
\newcommand{\tableSubItem}{\addtolength{\leftskip}{1em} \labelitemi \xspace}
\newcommand{\myHangIndent}{\hangindent=5mm}

%%%%%%%%%%%%%%%%%%%%%%%%%%%%%%%%%%%%%%%%%%%%%%%%%%%%%%%%%%%%%%
%%%%%%%%%%%%%%%%%%%%%%%%%%%%%%%%%%%%%%%%%%%%%%%%%%%%%%%%%%%%%%
\newcommand{\docTitle}{Curatorial Statement\xspace}
\begin{document}
\raggedright
\singlespacing

My duty as a collection-based scientist is to \textbf{\textit{advocate the use,
maintainence, and value of scientific research collections for understanding
the biodiversity of our planet.}}
A major challenge of curation is often striking a balance between the use and
long-term preservation of research material.
However, I believe we stand upon the precipice of a dramatic shift in how
biodiversity data is distributed and used by scientist that will greatly
reduce the negative impacts to the longevity of specimens.

\section*{The Future of Biodiversity Collections}
I believe we are very near a future where a typical museum specimen will be
associated with a  of digital information that will be available in real-time 
to scientist around the world.
geographic location, habitat, genome sequence, high-resolution, three-dimensional digital scan and x-ray.
While such rich digital informortion will never supplant the need for preserving
the physical specimen, I believe the data will truly revolutionize the role of
biodiversity collections in society.
The longevity of specimens will be extended, because most specimen-based
research can be conducted without ever using the
will greatly reduce the frequency at which the specimen is used.
If a specimen is ever damaged or lost blah
Such rich data will greatly expand the types of research that can be addressed.
Reduced use.
Backup.
justify the relevance to the public, funding agencies, and donors.
The collections that are at the forefront of embracing this shift will benefit

While this future may seem far off, the collections that are at the forefront
of embracing this shift will balh.

In my experience blah
there are many ways to begin this shift
make sure all current data is digital (genbank, treebase, dryad, citations,
digital photographs and audio recordings).
Digitizing in the field in realtime (photographs) to reduce information that is lost and corrupted along way

\section*{Collection-Based Research}
While my primary research questions focus on specific taxa \ldots
an important aspect of my work will always be broad biodiversity survey stuff.
Funding from NSF BSFI Nat Geo

The SNOMNH Herpetology collection is strong in blah
My research program would expand holdings of Old World blah, specifically
Southeast Asia.

\section*{Museum-Based Education}
Similar to curation, technology will greatly enhance museum-based education.

\end{document}
%%%%%%%%%%%%%%%%%%%%%%%%%%%%%%%%%%%%%%%%%%%%%%%%%%%%%%%%%%%%%%
%%%%%%%%%%%%%%%%%%%%%%%%%%%%%%%%%%%%%%%%%%%%%%%%%%%%%%%%%%%%%%
