\documentclass[10pt]{article}
%\usepackage{anysize}
%\papersize{11in}{8.5in}
%\marginsize{1in}{1in}{.5in}{.5in}
\textwidth = 6.5 in
\textheight = 9 in
\oddsidemargin = 0.0 in
\evensidemargin = 0.0 in
\topmargin = -0.5 in
\headheight = 0.35 in
\headsep = 0.15 in
\topskip = 0 in
\footskip = 0.5 in
\pagenumbering{arabic}
\usepackage{setspace}
\usepackage[usenames]{color}
\usepackage[fleqn]{amsmath}
\usepackage{graphicx}
\usepackage{url}
\usepackage{verbatim}
\usepackage{indentfirst}
\usepackage{booktabs}
\usepackage{multirow}
\usepackage[table]{xcolor}
\usepackage{ragged2e}
\usepackage{xspace}
\usepackage{parskip}
\usepackage{tabulary}
\usepackage[normalem]{ulem}
\usepackage[format=plain, labelsep=period, justification=raggedright, singlelinecheck=true, skip=2pt, font={footnotesize,sf}, labelfont=bf]{caption}
\usepackage{titlesec}
\usepackage{lastpage}
\usepackage{fancyhdr}
\usepackage{ifthen}
\usepackage{wrapfig}

%% Set up color palettes %%%%%%%%%%%%%%%%%%%%%%%%%%%%%%%%%%%%%%%%%%%%
% Color palette GreenOrange_6 from
% https://jiffyclub.github.io/palettable/tableau/
\definecolor{pgreen}     {RGB}{50,162,81}
\definecolor{porange}    {RGB}{255,127,15}
\definecolor{pblue}      {RGB}{60,183,204}
\definecolor{pyellow}    {RGB}{255,217,74}
\definecolor{pteal}      {RGB}{57,115,124}
\definecolor{pauburn}    {RGB}{184,90,13}
%%%%%%%%%%%%%%%%%%%%%%%%%%%%%%%%%%%%%%%%%%%%%%%%%%%%%%%%%%%%%%%%%%%%%
\definecolor{mygray}{gray}{0.9}

\usepackage{hyperref}
\hypersetup{pdfborder={0 0 0},
            colorlinks=true,
            % colorlinks=false,
            urlcolor=pteal,
            linkcolor=pteal,
            citecolor=pteal}
\usepackage{cleveref}
\crefformat{footnote}{#2\footnotemark[#1]#3}

%\usepackage[round]{natbib}
%\bibliographystyle{evolution}
% \usepackage[style=nature]{biblatex}
\usepackage[bibstyle=bib/joaks-statement,
    maxcitenames=3,
    mincitenames=2,
    backend=biber,
    bibencoding=utf8,
    date=year]{biblatex}

\newrobustcmd*{\shortfullcite}{\AtNextCite{\renewbibmacro{title}{}\renewbibmacro{in:}{}\renewbibmacro{number}{}}\fullcite}

\bibliography{references}

%% Format headers and footers %%%%%%%%%%%%%%%%%%%%
\pagestyle{fancy}
%\lhead{\ifthenelse{\value{page}=1}{}{\sffamily\footnotesize Jamie Oaks}}
\lhead{\sffamily \emph{\docTitle} \\ Jamie R.\ Oaks}
%\chead{\ifthenelse{\value{page}=1}{{\scshape \docTitle} \\ Jamie Richard Oaks}{\sffamily\footnotesize \docTitle}}
%\rhead{\ifthenelse{\value{page}=1}{}{\sffamily\footnotesize \today}}
\rhead{\sffamily \today}
\cfoot{\sffamily\footnotesize Page \thepage\ of \pageref{LastPage}}
\renewcommand{\headrulewidth}{0.4pt}
\renewcommand{\footrulewidth}{0pt}

%% Format section titles %%%%%%%%%%%%%%%%%%%%%%%%%
\titleformat{\section}[hang]
    {\large\sffamily\bfseries}
    {\S\ \thesection.}{.5em}{}[]
\titlespacing{\section}
    {0mm}{1.0ex plus .15ex minus .15ex}{0ex}

\titleformat{\subsection}[hang]
    {\large\sffamily\itshape}
    {\S\ \thesection.}{.5em}{}[]
\titlespacing{\subsection}
    {0mm}{0.5ex plus .1ex minus .1ex}{0ex}

\titleformat{\subsubsection}[runin]
    {\sffamily\bfseries}
    {\S\ \thesection.}{.5em}{}[.---]
\titlespacing{\subsubsection}
    {\parindent}{0.5ex plus .1ex minus .1ex}{0ex}

%% Format list environments %%%%%%%%%%%%%%%%%%%%%%%%
\renewcommand{\labelenumii}{\arabic{enumi}.\arabic{enumii}}
\renewcommand{\labelitemi}{$\circ$}

\newenvironment{myEnumerate}{
  \begin{enumerate}
    \setlength{\itemsep}{0.25em}
    \setlength{\parskip}{0pt}
    \setlength{\parsep}{0.5em}}
  {\end{enumerate}}

\newenvironment{myItemize}{
  \begin{itemize}
    \setlength{\leftskip}{-4mm}
    \setlength{\itemsep}{0.25em}
    \setlength{\parskip}{0pt}
    \setlength{\parsep}{0.5em}}
  {\end{itemize}}

\newenvironment{tightItemize}{%
\begin{itemize}[noitemsep, topsep=0pt, parsep=0pt, partopsep=0pt]}
{\end{itemize}}

\newenvironment{veryTightItemize}{%
\begin{itemize}[noitemsep, topsep=0pt, parsep=0pt, partopsep=0pt, leftmargin=*]}
{\end{itemize}}

\newenvironment{tightEnumerate}{%
\begin{enumerate}[noitemsep, topsep=0pt, parsep=0pt, partopsep=0pt]}
{\end{enumerate}}

\newenvironment{veryTightEnumerate}{%
\begin{enumerate}[noitemsep, topsep=0pt, parsep=0pt, partopsep=0pt, leftmargin=*]}
{\end{enumerate}}

%% Basic formatting and spacing %%%%%%%%%%%%%%%%%%%%%
\setlength{\parindent}{0em}
\setlength{\parskip}{0.5em}

%% My functions %%%%%%%%%%%%%%%%%%%%%%%%%%%%%
\newcommand{\ignore}[1]{}
\newcommand{\addTail}[1]{\textit{#1}.---}
\newcommand{\super}[1]{\ensuremath{^{\textrm{#1}}}}
\newcommand{\sub}[1]{\ensuremath{_{\textrm{#1}}}}
\newcommand{\dC}{\ensuremath{^\circ{\textrm{C}}}}
\newcommand{\tableSubItem}{\addtolength{\leftskip}{1em} \labelitemi \xspace}
\newcommand{\myHangIndent}{\hangindent=5mm}

%%%%%%%%%%%%%%%%%%%%%%%%%%%%%%%%%%%%%%%%%%%%%%%%%%%%%%%%%%%%%%
%%%%%%%%%%%%%%%%%%%%%%%%%%%%%%%%%%%%%%%%%%%%%%%%%%%%%%%%%%%%%%
\newcommand{\docTitle}{Curatorial Statement\xspace}
\begin{document}
\raggedright
\singlespacing

My duty as a collection-based scientist is to \textbf{\textit{advocate and
advance the use, maintenance, and value of natural history collections for
understanding the biodiversity of our planet.}}
The primary challenge of curation is striking a balance between the use and
long-term preservation of research material.
However, I believe we are at the beginning of a major shift in how
biodiversity data is distributed and used by scientists.
This shift will greatly mitigate this challenge and transform the role of
biodiversity collections
in biological research.

\section*{Curatorial Experience}
For the past 15 years, I have worked in the herpetology collections
of several natural history museums.
As a graduate student,
I served as the Curatorial Assistant (CA) for the herpetology collections
at the University of Kansas (KU) Biodiversity Institute and Louisiana State
University Museum of Natural Science (LSUMNS), the former of which is one of
the largest collections of amphibians and reptiles in the United States.
In the absence of a Collection Manager at both institutions, I was responsible
for all aspects of managing the collection, including specimen preparation,
accessions, cataloging, database management, loans, import/export permits, 
general collection and wet lab upkeep, and public tours.
Furthermore, while at KU, I made several improvements to the collection
database that increased the efficiency and accuracy of digitizing new
collection data.
As an NSF Postdoctoral Research Fellow, I helped prepare and accession
specimens for the Burke Museum Herpetology Collection.
Through my research, I contributed almost 450 specimens to these collections.

During my time at both KU and LSUMNS, we transferred the collection of genetic
resources into state-of-the-art, long-term liquid nitrogen storage facilities.
This gave me an appreciation for the importance of organization and database
management for keeping track of auxiliary preparation types.
I also experienced firsthand how integral collection improvement grants are
for the growth and long-term maintenance of biodiversity collections.

For the last three years, I have been the Curator of Amphibians and Reptiles at
the Auburn University Museum of Natural History (AUMNH).
While at the AUMNH,
I helped secure funding from the NSF Collections in Support of Biological
Research Program to install compactorized shelving in our wet collections.
We have just finished moving our specimens onto the new shelving systems, which
should allow our wet collections to continue to grow for the next two decades.
My lab has contributed approximately 630 specimens to the AUMNH herpetology and
genetic resources collections, including
240 from the Southeast Asia,
250 from the Southeastern United States (US),
and 140 from the Desert Southwest of the US.

\section*{Collection-Based Research}
All of the research in my lab is collection-based and is made possible by the
utility of natural history collections as biodiversity archives.
Our research into Southeast Asian biogeography requires targeted collecting of
particular taxa, as well as general collecting of amphibians and reptiles to
document the biodiversity of this understudied region of the world.
Accordingly, I will continue to fund such field work from extramural sources
such as the NSF and National Geographic Society.

One reason I am particularly excited about this position in the
Department of Biology
and Museum of Southwestern Biology (MSB)
at the University of New Mexico
is the prospect of developing
a local research program in the
Desert Southwest of the United States.
The rich diversity of amphibians and reptiles inhabiting the diverse
physiography of New Mexico offers an ideal template for studying evolutionary
processes in a spatially explicit and comparative framework.
My long-term research goals related to comparative phylogeography involve
interrogating functional genomic questions by using geographic sampling of
whole genomes from multiple taxa across a landscape.
In fact, one of my Ph.D.\ students, Randy Klabacka, is working on questions
related to the evolutionary costs of asexual reproduction using species of
parthenogenetic lizards across the Desert Southwest.
The University of New Mexico would provide a perfect base for this work, in
terms of colleagues, infrastructure, physical geography, and biodiversity.


\section*{The Future of Biodiversity Collections}
In the near future, a typical museum specimen will be associated
with a wealth of digital information that will be available in real-time
to scientists around the world.
In addition to the typical field data, there will be a genome sequence and a
high-resolution, three-dimensional digital scan for each specimen.
Thus, the entire genetic blueprint and internal and external morphology of 
each specimen will be a few keystrokes away from any computer in the world.
While such a rich digital representation will never supplant the need for
building and preserving collections of physical specimens, I believe the data
will truly revolutionize the role of biodiversity collections in science and
society.
By greatly reducing the frequency with which specimens need to be physically
transported and examined, the longevity of specimens will be extended.
Also, if a specimen is ever damaged or lost, the utility of the specimen will
be largely preserved in digital form.
Most importantly, having such rich data available from millions of specimens
across hundreds of institutions will greatly expand the types of research
questions that can be addressed with natural history collections.
Not only will this shift greatly benefit biodiversity science, it will also
increase the prominence of collections and clearly demonstrate their importance
to the public, policy-makers, funding agencies, and donors.


\section*{The Future is Now}
During my time working in natural history collections, the museum
community has made significant strides toward the seemingly distant
future described above.
Many natural history specimens are already associated with a wealth of
ancillary data that collection database software is getting ever better at
automatically consolidating,
updating, and serving to the scientific community in real-time.
For example, many specimens 
are associated with publications, sequences, and genomes on
\href{http://www.ncbi.nlm.nih.gov}{NCBI},
have digital photographs, video or audio recordings, and locality information
on platforms like
\href{https://www.inaturalist.org/}{iNaturalist},
are included in molecular, morphological, and ecological datasets deposited in
repositories on
\href{http://datadryad.org}{Dryad},
\href{https://zenodo.org/}{Zenodo},
or
\href{https://github.com/}{GitHub},
are associated with developmental or other data in
phenotype ontologies
like
\href{https://phenoscape.org/}{Phenoscape},
and are tips on phylogenetic trees in
\href{http://treebase.org}{TreeBASE} or on the
\href{http://opentreeoflife.org}{Open Tree of Life}.

Having worked with the developers of
\href{https://www.sustain.specifysoftware.org/}{Specify} while at KU to try to
streamline how ancillary data can be associated with specimens,
and having continued using Specify at the AUMNH,
I am keenly aware of its strengths and limitations.
I am very excited about the prospect of collaborating with colleagues at the
MSB who are at the forefront of open, community-based development of collection
informatics tools.
This would give me the opportunity to help contribute to the improvement of
biodiversity database software to help natural history collections serve as the
hub of information associated with their physical holdings.
% Collection databases need to continue to improve so that they can serve as the
% hub of information associated with the collection's physical holdings.
Two important aspects of this include (1) transforming how data is
collected and added to the database, and (2) adding flexibility and automation
to how ancillary data are incorporated and updated.

\subsubsection*{Transforming data collection}
Too often, data are lost or corrupted between field collection and database
entry.
Moving forward, I think one important solution to this problem will be the
development of portable, flexible, easy-to-use software that allows real-time
digitization of data as it is collected in the field.
The ultimate goal is to have software that can sync information such as GPS
coordinates, photographs, audio and video recordings, and micro-habitat and
other environmental data from wireless ``smart'' devices.
% We already have electronic devices that can collect much of this data and can
% communicate it via technologies such as Bluetooth.
% So, much of the infrastructure for this is already in place.
Not only will this make for richer and more accurate data associated with
voucher specimens, but it will also greatly increase the efficiency at which
new data is disseminated to the scientific community.

\subsubsection*{Automated updating of ancillary data}
Too often, valuable ancillary data never get associated with specimens.
It is impractical for collection staff to manually mine databases like NCBI,
Dryad, iNaturalist, and the Open Tree of Life in an effort to consolidate all
of the information available for every specimen.
Much of this work can be automated by data management systems to update the
ancillary data as it comes online.
% I am interested in developing flexible database software that automatically
% scans such databases and updates the ancillary data as it comes online.
Of course, the software should do this intelligently, flagging specimens
whenever there are conflicting data.
% This will transform collection databases into the most current and
% comprehensive source for all data associated with its holdings.
While the infrastructure for and curation of these data management systems
should be distributed across biodiversity collections, the information should
ultimately be served to the scientific community from standardized
web services with a rich application-programming interface (API).
Systems like
\href{https://arctosdb.org/}{Arctos},
\href{https://www.gbif.org/}{GBIF},
and
\href{https://www.idigbio.org/}{iDigBio}
are clear examples of the great progress that has been made in this direction,
and I am excited to see where we will be by the end of my career.




% \section*{Museum-Based Education}
% Similar to , 
% Similar to curation, technology will greatly enhance museum-based education.

\end{document}
%%%%%%%%%%%%%%%%%%%%%%%%%%%%%%%%%%%%%%%%%%%%%%%%%%%%%%%%%%%%%%
%%%%%%%%%%%%%%%%%%%%%%%%%%%%%%%%%%%%%%%%%%%%%%%%%%%%%%%%%%%%%%
