\documentclass[10pt]{article}
%\usepackage{anysize}
%\papersize{11in}{8.5in}
%\marginsize{1in}{1in}{.5in}{.5in}
\textwidth = 6.5 in
\textheight = 9 in
\oddsidemargin = 0.0 in
\evensidemargin = 0.0 in
\topmargin = -0.5 in
\headheight = 0.35 in
\headsep = 0.15 in
\topskip = 0 in
\footskip = 0.5 in
\pagenumbering{arabic}
\usepackage{setspace}
\usepackage[usenames]{color}
\usepackage[fleqn]{amsmath}
\usepackage{graphicx}
\usepackage{url}
\usepackage{verbatim}
\usepackage{indentfirst}
\usepackage{booktabs}
\usepackage{multirow}
\usepackage[table]{xcolor}
\usepackage{ragged2e}
\usepackage{xspace}
\usepackage{parskip}
\usepackage{tabulary}
\usepackage[normalem]{ulem}
\usepackage[format=plain, labelsep=period, justification=raggedright, singlelinecheck=true, skip=2pt, font={footnotesize,sf}, labelfont=bf]{caption}
\usepackage{titlesec}
\usepackage{lastpage}
\usepackage{fancyhdr}
\usepackage{ifthen}
\usepackage{wrapfig}

%% Set up color palettes %%%%%%%%%%%%%%%%%%%%%%%%%%%%%%%%%%%%%%%%%%%%
% Color palette GreenOrange_6 from
% https://jiffyclub.github.io/palettable/tableau/
\definecolor{pgreen}     {RGB}{50,162,81}
\definecolor{porange}    {RGB}{255,127,15}
\definecolor{pblue}      {RGB}{60,183,204}
\definecolor{pyellow}    {RGB}{255,217,74}
\definecolor{pteal}      {RGB}{57,115,124}
\definecolor{pauburn}    {RGB}{184,90,13}
%%%%%%%%%%%%%%%%%%%%%%%%%%%%%%%%%%%%%%%%%%%%%%%%%%%%%%%%%%%%%%%%%%%%%
\definecolor{mygray}{gray}{0.9}

\usepackage{hyperref}
\hypersetup{pdfborder={0 0 0},
            colorlinks=true,
            % colorlinks=false,
            urlcolor=pteal,
            linkcolor=pteal,
            citecolor=pteal}
\usepackage{cleveref}
\crefformat{footnote}{#2\footnotemark[#1]#3}

%\usepackage[round]{natbib}
%\bibliographystyle{evolution}
% \usepackage[style=nature]{biblatex}
\usepackage[bibstyle=bib/joaks-statement,
    maxcitenames=3,
    mincitenames=2,
    backend=biber,
    bibencoding=utf8,
    date=year]{biblatex}

\newrobustcmd*{\shortfullcite}{\AtNextCite{\renewbibmacro{title}{}\renewbibmacro{in:}{}\renewbibmacro{number}{}}\fullcite}

\bibliography{references}

%% Format headers and footers %%%%%%%%%%%%%%%%%%%%
\pagestyle{fancy}
%\lhead{\ifthenelse{\value{page}=1}{}{\sffamily\footnotesize Jamie Oaks}}
\lhead{\sffamily \emph{\docTitle} \\ Jamie R.\ Oaks}
%\chead{\ifthenelse{\value{page}=1}{{\scshape \docTitle} \\ Jamie Richard Oaks}{\sffamily\footnotesize \docTitle}}
%\rhead{\ifthenelse{\value{page}=1}{}{\sffamily\footnotesize \today}}
\rhead{\sffamily \today}
\cfoot{\sffamily\footnotesize Page \thepage\ of \pageref{LastPage}}
\renewcommand{\headrulewidth}{0.4pt}
\renewcommand{\footrulewidth}{0pt}

%% Format section titles %%%%%%%%%%%%%%%%%%%%%%%%%
\titleformat{\section}[hang]
    {\large\sffamily\bfseries}
    {\S\ \thesection.}{.5em}{}[]
\titlespacing{\section}
    {0mm}{1.0ex plus .15ex minus .15ex}{0ex}

\titleformat{\subsection}[hang]
    {\large\sffamily\itshape}
    {\S\ \thesection.}{.5em}{}[]
\titlespacing{\subsection}
    {0mm}{0.5ex plus .1ex minus .1ex}{0ex}

\titleformat{\subsubsection}[runin]
    {\sffamily\bfseries}
    {\S\ \thesection.}{.5em}{}[.---]
\titlespacing{\subsubsection}
    {\parindent}{0.5ex plus .1ex minus .1ex}{0ex}

%% Format list environments %%%%%%%%%%%%%%%%%%%%%%%%
\renewcommand{\labelenumii}{\arabic{enumi}.\arabic{enumii}}
\renewcommand{\labelitemi}{$\circ$}

\newenvironment{myEnumerate}{
  \begin{enumerate}
    \setlength{\itemsep}{0.25em}
    \setlength{\parskip}{0pt}
    \setlength{\parsep}{0.5em}}
  {\end{enumerate}}

\newenvironment{myItemize}{
  \begin{itemize}
    \setlength{\leftskip}{-4mm}
    \setlength{\itemsep}{0.25em}
    \setlength{\parskip}{0pt}
    \setlength{\parsep}{0.5em}}
  {\end{itemize}}

\newenvironment{tightItemize}{%
\begin{itemize}[noitemsep, topsep=0pt, parsep=0pt, partopsep=0pt]}
{\end{itemize}}

\newenvironment{veryTightItemize}{%
\begin{itemize}[noitemsep, topsep=0pt, parsep=0pt, partopsep=0pt, leftmargin=*]}
{\end{itemize}}

\newenvironment{tightEnumerate}{%
\begin{enumerate}[noitemsep, topsep=0pt, parsep=0pt, partopsep=0pt]}
{\end{enumerate}}

\newenvironment{veryTightEnumerate}{%
\begin{enumerate}[noitemsep, topsep=0pt, parsep=0pt, partopsep=0pt, leftmargin=*]}
{\end{enumerate}}

%% Basic formatting and spacing %%%%%%%%%%%%%%%%%%%%%
\setlength{\parindent}{0em}
\setlength{\parskip}{0.5em}

%% My functions %%%%%%%%%%%%%%%%%%%%%%%%%%%%%
\newcommand{\ignore}[1]{}
\newcommand{\addTail}[1]{\textit{#1}.---}
\newcommand{\super}[1]{\ensuremath{^{\textrm{#1}}}}
\newcommand{\sub}[1]{\ensuremath{_{\textrm{#1}}}}
\newcommand{\dC}{\ensuremath{^\circ{\textrm{C}}}}
\newcommand{\tableSubItem}{\addtolength{\leftskip}{1em} \labelitemi \xspace}
\newcommand{\myHangIndent}{\hangindent=5mm}

%%%%%%%%%%%%%%%%%%%%%%%%%%%%%%%%%%%%%%%%%%%%%%%%%%%%%%%%%%%%%%
%%%%%%%%%%%%%%%%%%%%%%%%%%%%%%%%%%%%%%%%%%%%%%%%%%%%%%%%%%%%%%
\newcommand{\docTitle}{Curatorial Statement\xspace}
\begin{document}
\raggedright
\singlespacing

My duty as a collection-based scientist is to \textbf{\textit{advocate and
advance the use, maintainence, and value of natural history collections for
understanding the biodiversity of our planet.}}
The primary challenge of curation is striking a balance between the use and
long-term preservation of research material.
However, I believe we are at the beginning of a major shift in how
biodiversity data is distributed and used by scientists.
This shift will greatly mitigate this challenge and transform the role of
biodiversity collections.

\section*{The Future of Biodiversity Collections}
In the near future, a typical museum specimen will be associated
with a wealth of digital information that will be available in real-time
to scientists around the world.
In addition to the typical field data, there will be a genome sequence
and a high-resolution, three-dimensional digital scan and x-ray for each
specimen.
Thus, the entire genetic blueprint and internal and external morphology of 
each specimen will be a few keystrokes away from any computer in the world.
While such a rich digital representation will never supplant the need for
building and preserving collections of physical specimens, I believe the data
will truly revolutionize the role of biodiversity collections in science and
society.
By greatly reducing the frequency with which specimens need be physically
transported and examined, the longevity of specimens will be extended.
Also, if a specimen is ever damaged or lost, the utility of the specimen will
be largely preserved in digital form.
Most importantly, having such rich data available from millions of specimens
across hundreds of institutions will greatly expand the types of research
questions that can be addressed with natural history collections.
Not only will this shift greatly benefit biodiversity science, it will also
increase the prominence of collections and clearly demonstrate their importance
to the public, policy-makers, funding agencies, and donors.


\section*{The Future is Now}
The future described above may seem distant.
However, during my experience working in natural history collections, I have
realized that there are many significant steps that need to be taken now.
Many natural history specimens are already associated with a wealth of ancillary
data that collection databases are ill-equipped to automatically consolidate,
update, and serve to the community in real-time.
For example, many specimens 
have digital photographs and/or audio recordings,
are associated with publications and sequences on
\href{http://www.ncbi.nlm.nih.gov}{NCBI}, are included in molecular and
morphological datasets deposited in repositories like
\href{http://datadryad.org}{Dryad}, are tips on phylogenetic trees in
\href{http://treebase.org}{TreeBASE}, or better yet, on the
\href{http://opentreeoflife.org}{Open Tree of Life}.
Collection databases need to be modernized so that they can serve
as the central hub of information associated with the collection's
physical holdings.

Two steps are needed to accomplish this: (1) transforming how data is
collected and added to the database, and (2) adding flexibility and automation
to how ancillary data are incorporated and updated.

\subsubsection*{Transforming data collection}
Too much data is lost or corrupted between field collection and database entry.
I am interested in developing portable, flexible, easy-to-use software
that allows real-time digitization of data as it is collected in the field.
The ultimate goal is to have software that can sync information such as
GPS coordinates, photographs, audio-recordings, micro-habitat data, and
weather data from wireless devices.
% We already have electronic devices that can collect much of this data and can
% communicate it via technologies such as Bluetooth.
% So, much of the infrastructure for this is already in place.
Not only will this make for richer and more accurate data associated with
vouchers, but will also greatly increase the efficiency at which new
data is disseminated to the scientific community.

\subsubsection*{Automated updating of ancillary data}
Too often valuable ancillary data never gets associated with specimens.
It is impractical for collection staff to manually mine databases like NCBI,
GenBank, Dryad, and TreeBASE in an effort to consolidate all of the information
available for every specimen.
I am interested in developing flexible database software that automatically
scans such databases and updates the ancillary data as it comes online.
The software should do this intelligently, flagging specimens whenever there
are conflicting data (e.g., a phylogenetic placement conflicts with the current
taxonomy).
This will transform the collection database into the most current and
comprehensive source for all data associated with its holdings.
This, after all, should be the goal of all collections.

% \section*{Museum-Based Education}
% Similar to , 
% Similar to curation, technology will greatly enhance museum-based education.

\section*{Collection-Management Experience}
I have served as the Curatorial Assistant (CA) for the Herpetology collections
at the University of Kansas (KU) Biodiversity Institute and Louisiana State
University Museum of Natural Science (LSUMNS), the former of which is one of
the largest collections of amphibians and reptiles in the United States.
In the absence of a Collection Manager at both institutions, I was responsible
for all aspects of managing the collection, including specimen preparation,
accessions, cataloging, database management, loans, import/export permits, 
general collection and wet lab upkeep, and public tours.
Furthermore, while at KU, I have made several improvements to the collection
database that have increased the efficiency and accuracy of digitizing new
collection data.

During my time at both KU and LSUMNS, we transferred the collection of tissue
aliquots into state-of-the-art, long-term liquid nitrogen storage facilities.
This gave me an appreciation for the importance of organization and database
management for keeping track of auxiliary preparation types.
I also experienced first hand how integral collection improvement grants are
for the growth and long-term maintenance of biodiversity collections.

\section*{Collection-Based Research}
All of my empirical research is collection-based.
Without the utility of natural history collections as biodiversity archives,
my work would not be possible.
My primary research questions require targeted collecting of particular
Southeast Asian taxa.
An equally important aspect of my research program entails general 
collecting of reptiles and amphibians to document the biodiversity of
this understudied region of the world.
Accordingly, I intend to fund such survey work
from sources such as the  NSF Biodiversity Inventories
Program and National Geographic Society.

The Herpetology collection at the Sam Noble Oklahoma Museum of Natural History
has impressive holdings of specimens from the United States and the Neotropics.
My research program would diversify the collection and strengthen Old World
sampling, particularly from Southeast Asia.

\end{document}
%%%%%%%%%%%%%%%%%%%%%%%%%%%%%%%%%%%%%%%%%%%%%%%%%%%%%%%%%%%%%%
%%%%%%%%%%%%%%%%%%%%%%%%%%%%%%%%%%%%%%%%%%%%%%%%%%%%%%%%%%%%%%

